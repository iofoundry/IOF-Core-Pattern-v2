\chapter{Measurement}

\section{Simple vs Aggregate Measurement}
\subsection*{Scenario Objective}


This scenario demonstrates how to represent measurements within the context of the \textbf{BFO/IOF ontology}. It then also represents how to capture processing simple measurements to get an aggregate result (such as doing an average). 

\subsection*{Key Points}
\begin{itemize}
    \item This pattern highlights the connection between a measurement and the attribute that is being measured
    \item The pattern demonstrates how measurements are processed and introduces a distinction between measured data and process data and its association with an attribute
    \item For the examples in hand QUDT is used for representing units according to the guide x. It should be noted that using QUDT is not normative for the IOF
\end{itemize}

\subsection*{General Pattern Description} 
\subsection{Use case: Measuring protein concentration with a  Bicinchoninic Acid (BCA) Assay}

\section{Unitless measurements}
count, percentage. ratio

\section{Multiple measurements of the same object at the same time}
different measurement processes 
measurements using different instruments

\section{Time-series measurement}
(process characteristics)

\section{uncertainty, range of values}
if time permits

