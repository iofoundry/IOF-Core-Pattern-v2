%%\chapter{Processes}
\label{chapter-scenario-processes}

\section*{Simple Process Sequence}

\textbf{Created by:} Jim Logan \\
\textbf{Modified by:} Jim Logan \\

\subsection*{Scenario Objective}
% Instruction box
This scenario demonstrates how to represent a sequence of time stamped events within the context of the \textbf{BFO/IOF ontology}. A later scenario will demonstrate how to automatically classify an overall process based on one step following another.

\textit{ 
Describe the general purpose of representing the pattern (e.g., demonstrating how you can depict measurements with IOF core).}

\subsection*{General Pattern Description}

\textit{  
Provide a figure containing the pattern independent of the use case (e.g., just measurements in general as opposed to measuring a particular thing such as pH or length). \\
\noindent \textit{This diagram is generally a simple class-relation diagram. Please see examples of class-relation diagram here.}
  }

\textit{ 
Describe the general pattern, including any background why such types and relations are used. Also mention any alternative or shortcuts. Please provide a background of the types and relations in the context of IOF (please prefer logical arguments to didactic arguments).
}


\subsection*{Use Case: <Use-Case title>}
 \textit{ 
Describe a concrete use case for the pattern. (Multiple use cases can be described for one pattern. If multiple, please use distinct use case titles and repeat all subsubsections for each use case.)
  }

\subsubsection*{Use-Case Pattern Description}
 \textit{ 
Provide a diagram which matches the use-case pattern description. \\
\noindent \textit{This diagram is generally an object diagram. Please see examples of object diagram here.}
  }

 \textit{ 
Describe how the general pattern is applied to the use case. It is important to highlight within the description how the use-case-specific concepts align with the general pattern (e.g., SubClassOf Class for general pattern).
  }

\subsubsection*{Use-Case Example Data}
 \textit{ 
Provide a description of the data set used to demonstrate the use case pattern (e.g., format as JSON, CSV, XML, column/attribute/node name and their description, relation between datasets if multiple are used). Use not less than 3 and not more than 7 records/transactions.
  }
    
\csvautotabular{data/EventLog.csv} 
\begin{tabular}{l|l}%
    \textbf{Time (s)} & \textbf{Zeroed time (s)}% specify table head
    \csvreader[head to column names]{data/EventLog.csv}{}% use head of csv as column names

\end{tabular}

\subsubsection*{Data Mapping}
 \textit{ 
Describe how the data was mapped to RDF. Provide an INSERT DATA/ INSERT SPARQL for mapping. Use INSERT only when the use case example data needs further manipulation. \\
For INSERT DATA SPARQL, use only 2/3 records/transactions with named class and individuals. \\
For INSERT SPARQL, declare column names by `\{ \}' to the variable.  
For INSERT query, 
For both, do not use blank nodes.    
  }

\subsubsection*{Data Validation}
 \textit{ 
Data validation can be performed in two ways: accessing interesting facts using SPARQL or validating whether the entire data conforms to the ontology using SHACL. It is preferable to provide both. \\
Provide the SPARQL query in the code block along with the result of the query. \\
  }