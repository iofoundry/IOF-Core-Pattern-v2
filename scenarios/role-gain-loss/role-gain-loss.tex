\section{Gain and loss of roles}

\textbf{Created by:} Arkopaul Sarkar \\
\textbf{Modified by:} Arkopaul Sarkar \\

\subsection*{Scenario Objective}

BFO roles are realizable entities external to the bearer and assigned within specific physical, social, or institutional contexts. A bearer can hold multiple roles over time or simultaneously, without ceasing to exist when roles change. However, BFO lacks constructs to indicate when roles begin or end. This scenario discussed the patterns around two subclasses of bfo:process, introduced by IOF Core to represent the the start and end of role. 

\subsection*{General Pattern Description}



\includesvg[scale=0.6]{scenarios/role-gain-loss/images/general-pattern.svg}


\subsection*{Use Case: <Use-Case title>}
 \textit{ 
Describe a concrete use case for the pattern. (Multiple use cases can be described for one pattern. If multiple, please use distinct use case titles and repeat all subsubsections for each use case.)
  }

\subsubsection*{Use-Case Pattern Description}
 \textit{ 
Provide a diagram which matches the use-case pattern description. \\
\noindent \textit{This diagram is generally an object diagram. Please see examples of object diagram here.}
  }

 \textit{ 
Describe how the general pattern is applied to the use case. It is important to highlight within the description how the use-case-specific concepts align with the general pattern (e.g., SubClassOf Class for general pattern).
  }

\subsubsection*{Use-Case Example Data}
 \textit{ 
Provide a description of the data set used to demonstrate the use case pattern (e.g., format as JSON, CSV, XML, column/attribute/node name and their description, relation between datasets if multiple are used). Use not less than 3 and not more than 7 records/transactions.
  }

\subsubsection*{Data Mapping}
 \textit{ 
Describe how the data was mapped to RDF. Provide an INSERT DATA/ INSERT SPARQL for mapping. Use INSERT only when the use case example data needs further manipulation. \\
For INSERT DATA SPARQL, use only 2/3 records/transactions with named class and individuals. \\
For INSERT SPARQL, declare column names by `\{ \}' to the variable.  
For INSERT query, 
For both, do not use blank nodes.    
  }

\subsubsection*{Data Validation}
 \textit{ 
Data validation can be performed in two ways: accessing interesting facts using SPARQL or validating whether the entire data conforms to the ontology using SHACL. It is preferable to provide both. \\
Provide the SPARQL query in the code block along with the result of the query. \\
  }